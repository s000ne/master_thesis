This section explains the detailed character of the spatial momentum of a free-floating base robot.
In particular, angular momentum varies depending on the reference coordinate frame.
With space robots, the base coordinate frame has been made use of representing the angular momentum.
On the other hand, the angular momentum with respect to the system CoM can be feasible
under controlling humanoid robots.
We describe the relation of these momenta in what follows.

%%%%%%%%%%%%%%%%%%%%%%%%%%%%%%%%%%%%
\section{Spatial transformation}
%%%%%%%%%%%%%%%%%%%%%%%%%%%%%%%%%%%%
Let us consider a free-floating base robot that consists of $n$ joints manipulator(s)
and a floating base.
Then, the most basic definition of linear and angular momentum is expressed as follows:
% 
% ---------------------------------------------------------------------
\begin{align}
  \bm{p} &= \sum_{i=0}^{n}m_{i}\bm{v}_{i} = m_{C}\bm{v}_{C}\label{eq:APP_LINEAR_MOM}\\
  \bm{l}_{F} &= \sum_{i=0}^{n}\Big\{\bm{I}_{i}\bm{\omega}_{i} + m_{i}[\bm{r}_{i}^{\times}]\bm{v}_{i}\Big\}\label{eq:APP_ANG_MOM_F}
\end{align} 
% ---------------------------------------------------------------------
%
where $\bm{l}_{F}\R{3}$ is angular momentum with respect to the inertial frame $\{F\}$.
$\bm{r}_{i}$, $\bm{v}_{i}$ and $\bm{\omega}_{i}\R{3}$ stand for
position vector pointing to the center of mass (CoM) of each link,
linear velocity of each link CoM and
angular velocity of each link, respectively.
$m_{i}$, $\bm{I}_{i}\R{3 \times 3}$ are mass and inertia tensor of each link around
the center of mass of each link.
Note that the inertia tensor is described with respect to the inertial frame.

If the position vector is expressed as $\bm{r}_{i} = \bm{r}_{C} + \bm{r}_{C \rightarrow i}$,
where $\bm{r}_{C}$ and $\bm{r}_{C \rightarrow i}$ are position vector pointing to the system CoM and
each link CoM with respect to the system CoM,
the angular momentum can be rewritten as follows:
%
\begin{align}\label{eq:LF}
  \bm{l}_{F} = &\sum_{i=0}^{n}\Big\{\bm{I}_{i}\bm{\omega}_{i}\Big\} + 
  \sum_{i=0}^{n}\Big\{m_{i}(\bm{r}_{C} + \bm{r}_{C \rightarrow i})\times (\bm{v}_{C} + \bm{v}_{C \rightarrow i})\Big\}\\
  = &\sum_{i=0}^{n}\Big\{\bm{I}_{i}\bm{\omega}_{i}\Big\}
  + \sum_{i=0}^{n}m_{i}[\bm{r}_{C}^{\times}]\bm{v}_{C} + \sum_{i=0}^{n}m_{i}[\bm{r}_{C}^{\times}]\bm{v}_{C \rightarrow i}\notag\\
  &+ \sum_{i=0}^{n}m_{i}[\bm{r}_{C \rightarrow i}^{\times}]\bm{v}_{C} + \sum_{i=0}^{n}m_{i}[\bm{r}_{C \rightarrow i}^{\times}]\bm{v}_{C \rightarrow i}\notag\\
  = & \sum_{i=0}^{n}\Big\{\bm{I}_{i}\bm{\omega}_{i}\Big\} + [\bm{r}_{C}^{\times}]\bm{p} +
  \sum_{i=0}^{n}[\bm{r}_{C \rightarrow i}^{\times}]\bm{v}_{C \rightarrow i}\notag
\end{align}
%
where $\bm{v}_{C}\R{3}$ denotes linear velocity of the system CoM.
Note that we made use of the conditions,
$\sum_{i=0}^{n}m_{i}[\bm{r}_{C \rightarrow i}^{\times}] = \bm{0}$ and
$\sum_{i=0}^{n}m_{i}[\bm{v}_{C \rightarrow i}^{\times}] = \bm{0}$, in the above formulation.

Next, we consider the case when all joints are locked ($\th = \bm{0}$).
Then, the following relation can be obtained:
%
\begin{align}
  \bm{\omega}_{i} &= \bm{\omega}_{C}\\
  \bm{v}_{C \rightarrow i} &= [\bm{r}_{i \rightarrow C}^{\times}]\bm{\omega}_{C}
\end{align}
%
Substituting these relations into \eq{LF},
we can obtain the following equation:
%
\begin{align}
  \bm{l}_{F} &= \sum_{i=0}^{n}\bm{I}_{i}\bm{\omega}_{C} + \bm{r}_{C} \times \bm{p}
  + \sum_{i=0}^{n}m_{i}[\bm{r}_{C \rightarrow i}^{\times}][\bm{r}_{i \rightarrow C}^{\times}]\bm{\omega}_{C}\\
  &= \sum_{i=0}^{n}\Big\{\bm{I}_{i} + m_{i}[\bm{r}_{C \rightarrow i}^{\times}][\bm{r}_{i \rightarrow C}^{\times}]\Big\}\bm{\omega}_{C} + \bm{r}_{C}\times \bm{p}\notag\\
  &= \bm{I}_{C}\bm{\omega}_{C} + \bm{r}_{C}\times \bm{p}\notag\label{eq:APP_ANG_MOM_F2}
\end{align}
%
where $\bm{I}_{C} = \sum_{i=0}^{n}\{\bm{I}_{i} + m_{i}[\bm{r}_{C \rightarrow i}^{\times}][\bm{r}_{i \rightarrow C}^{\times}]\}\R{3 \times 3}$ is
the CRB inertia tensor when all joint are locked.

On the other hand, the angular momentum with respect to the CoM $\{C\}$ can be obtained through
spatial transformation from $\{F\}$ as:
%
\begin{align}
  \mathcal{L}_{C} &= {}^{C}\bm{T}_{F}^{T}\mathcal{L}_{F}
\end{align}
%
or
%
\begin{align}
  \mathcal{L}_{C} 
  = \bmat{\bm{p} \\ \bm{l}_{C}} = \bmat{\bm{E} & \bm{0} \\ -[\bm{r}_{C}^{\times}] & \bm{E}}\bmat{\bm{p} \\ \bm{l}_{F}}
  = \bmat{\bm{p} \\ \bm{l}_{F}} - \bmat{\bm{0}\\\bm{r}_{C}\times \bm{p}}\label{eq:APP_ANG_MOM_C}
\end{align}
%
where $\mathcal{L}_{i}\R{6}$ denotes spatial momentum.

According to the equations \eq{APP_LINEAR_MOM} and \eq{APP_ANG_MOM_C},
we can rewrite the spatial momentum with respect to $\{C\}$ as follows:
%
\begin{align}
  \mathcal{L}_{C} &= \bm{M}_{C}\mathcal{V}_{C}\\
  &= \bmat{m_{C}\bm{E} & \bm{0} \\ \bm{0} & \bm{I}_{C}}\bmat{\bm{v}_{C}\\\bm{\omega}_{C}}\label{eq:APP_LC}.
\end{align}
%

On the other hand,
the spatial momentum with respect to the base CoM $\mathcal{L}_{B}$ can be obtained
through a spatial transformation from $\mathcal{L}_{C}$ as follows:
%
\begin{align}
  \mathcal{L}_{B} = {}^{B}\bm{T}_{C}^{T}\mathcal{L}_{C}\label{eq:APP_LB_LC}
\end{align}
%
or
%
\begin{align}
  \mathcal{L}_{B} = \bmat{\bm{p}\\ \bm{l}_{B}} = \bmat{\bm{p} \\ \bm{l}_{C} - [\bm{r}_{C \rightarrow B}^\times]\bm{p}}
\end{align}
%

%%%%%%%%%%%%%%%%%%%%%%%%%%%%%%%%%%
\section{Spatial CoM velocity}
%%%%%%%%%%%%%%%%%%%%%%%%%%%%%%%%%%
We derive the spatial velocity of the system CoM without considering locked joints ($\th \not = \bm{0}$).
First, the linear velocity of the system CoM is defined as follows:
%
\begin{align}
  \bm{v}_{C} &= \frac{1}{m_{C}}\sum_{i=0}^{n}m_{i}\bm{v}_{i}\\
  &= \bm{v}_{b} - [\bm{r}_{b \rightarrow C}^{\times}]\bm{\omega}_{b} + \bm{J}_{C}\thd
\end{align}
%
where $\bm{J}_{C}\R{3 \times n}$ denotes the system CoM Jacobian.

On the other hand,
the angular velocity of the system CoM cannot be obtained by such straightforward manner.
In order to derive it, we first obtain the spatial momentum with respect to the system CoM
in terms of spatial base velocity and joint velocity.
The base spatial momentum $\mathcal{L}_{B}$ is expressed in the following form \cite{Masutani}:
%
\begin{align}
  \mathcal{L}_{B} =
  \begin{bmatrix}
    m_{C}\bm{E} & m_{C}[\bm{r}_{b \rightarrow C}^{\times}]^{T} & m_{C}\bm{J}_{C}\\
    m_{C}[\bm{r}_{b \rightarrow C}^{\times}] & \bm{M}_{\omega} & \bm{M}_{\omega l}
  \end{bmatrix}
  \bmat{\bm{v}_{b} \\ \bm{\omega}_{b} \\ \thd}
\end{align}
%
Through the transformation in \eq{APP_LB_LC},
the CoM spatial momentum can be obtained as follows:
%
\begin{align}
  \mathcal{L}_{C} &= {}^{C}\bm{T}_{B}^{-1}\mathcal{L}_{B}\label{eq:APP_LC2}\\
  &=
  \begin{bmatrix}
    m_{C}\bm{E} & m_{C}[\bm{r}_{b \rightarrow C}^{\times}]^{T} & m_{C}\bm{J}_{C}\\
    \bm{0} & \bm{I}_{C} & \bm{H}_{C}
  \end{bmatrix}
  \bmat{\bm{v}_{b} \\ \bm{\omega}_{b} \\ \thd}\notag\\
  \bm{I}_{C} &= \sum_{i=0}^{n}\bm{I}_{i} + \sum_{i=1}^{n}m_{i}[\bm{r}_{C \rightarrow i}^{\times}][\bm{r}_{i \rightarrow b}^{\times}]\label{eq:APP_IC}\\
  \bm{H}_{C} &= \sum_{i=1}^{n}\{\bm{I}_{i}\bm{J}_{\omega_{i}} + m_{i}[\bm{r}_{C \rightarrow i}^{\times}]\bm{J}_{v_{i}}\}
\end{align}
%
Note that the definition of $\bm{I}_{C}$ seems to be difference from that in \eq{APP_ANG_MOM_F2}.
However, these are equivalent as seen from the following proof.

First, we rewrite the CRB inertia tensor described in \eq{APP_ANG_MOM_F2} into the following form:
%
\begin{align}
  \bm{I}_{C} &= \sum_{i=0}^{n}\{\bm{I}_{i} + m_{i}[\bm{r}_{C \rightarrow i}^{\times}]^{T}[\bm{r}_{C \rightarrow i}^{\times}]\\
  &= \sum_{i=0}^{n}\bm{I}_{i} + \sum_{i=1}^{n}m_{i}[\bm{r}_{C \rightarrow i}^{\times}][\bm{r}_{i \rightarrow C}^{\times}] +
  m_{b}[\bm{r}_{C \rightarrow b}^{\times}][\bm{r}_{b \rightarrow C}^{\times}]\label{eq:APP_IC2}
\end{align}
%
Then, subtracting \eq{APP_IC2} from \eq{APP_IC},
we obtain:
%
\begin{align}
  &\sum_{i=1}^{n}m_{i}[\bm{r}_{C \rightarrow i}^{\times}][{}^{i}\bm{r}_{C}^{\times}] + m_{b}[\bm{r}_{C \rightarrow b}^{\times}][\bm{r}_{b \rightarrow C}^{\times}]
  - \sum_{i=1}^{n}m_{i}[\bm{r}_{C \rightarrow i}^{\times}][\bm{r}_{i \rightarrow b}^{\times}]\notag\\
  = &\sum_{i=1}^{n}m_{i}[\bm{r}_{C \rightarrow i}^{\times}][\bm{r}_{b \rightarrow C}^{\times}]
  + m_{b}[\bm{r}_{C \rightarrow b}^{\times}][\bm{r}_{b \rightarrow C}^{\times}]\notag\\
  = &\Big\{\sum_{i=0}^{n}m_{i}[\bm{r}_{C \rightarrow i}^{\times}]\Big\}[\bm{r}_{b \rightarrow C}^{\times}] \notag\\
  = &\bm{0}
\end{align}
%
Note that we made use of the condition $\sum m_{i}[\bm{r}_{C \rightarrow i}^{\times}] = \bm{0}$;
this implies that the CoM position vector with respect to the CoM, i.e.\ zero.

Finally, we can derive the CoM angular velocity from \eq{APP_IC} and \eq{APP_IC2}.
According to these equations, the angular momentum with respect to the system CoM can be expressed as follows:
%
\begin{align}
  \bm{l}_{C} &= \bm{I}_{C}\bm{\omega}_{C}\\
  &= \bm{I}_{C}\bm{\omega}_{b} + \bm{H}_{C}\thd
\end{align}
%
Hence, the CoM angular velocity can be derived as:
%
\begin{align}
  \bm{\omega}_{C} = \bm{\omega}_{b} + \bm{I}_{C}^{-1}\bm{H}_{C}\thd
\end{align}
%
Then, the spatial velocity of the system CoM is written as:
%
\begin{align}
  \mathcal{V}_{C} = {}^{C}\bm{T}_{B}\mathcal{V}_{b} + \bmat{\bm{J}_{C} \\ \bm{I}_{C}^{-1}\bm{H}_{C}}\thd.
\end{align}
%






%**********************************************************************
%
%
%%% Local Variables:
%%% mode: latex
%%% TeX-master: "../main"
%%% End:
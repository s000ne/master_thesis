% Abstract for this chapter
%
%**********************************************************************

%%%%%%%%%%%%%%%%%%%%%%%
\section{Conclusions}
%%%%%%%%%%%%%%%%%%%%%%%
The aim of this work was developing control methods based on the Reaction Null-Space formulation
for redundant manipulators.
In particular, we tackled two control issues:
(i) reactionless motion control for free-floating space robots and
(ii) motion/force control for fixed-base redundant manipulators.

In the case of the first issue,
we aimed to propose some tasks suitable for execution under reactionless motion control.
First, we examined the properties of reactionless motion with a planar two-DoF model.
Through vector field analysis,
we confirmed that the manipulator attachment position plays an important role as a bifurcation parameter.
Then, the reactionless motions of a seven-DoF redundant manipulator was discussed.
From the perspective of the kinematic structure of the model,
we represent reactionless motion of the model as superposition of the
predominant wrist and elbow motion.
Based on these motions,
we proposed three motion tasks:
(a) an inspection task using a hand camera,
(b) a point-to-point positioning task and
(c) a deployment task from a stowed configuration.
The performances were verified through numerical simulation.

As an interesting feature of reactionless motion control,
we found out that reactionless motion coincides with instantaneous minimum energy motion
when no base-attitude deviation is assumed.
This feature arises from the high energy consumption of reaction wheels.
The kinetic energy to compensate the base reaction induced by manipulator motion is represented
as a quadratic function of the manipulator parameters,
while the kinetic energy of manipulator is a linear function.
Hence, reactionless motion is feasible from the perspective of energy efficiency due to the no usage of
reaction wheels.
This feature was verified via numerical calculation.
Finally, we compared the energy consumption of reactionless motion with
that of reaction wheels used control system,
under the inspection task explained above.
From the above results,
we can conclude that reactionless motion control is feasible from the perspective of both time and
energy efficiency.

On the other hand,
we proposed motion/force control based on the Reaction Null-Space formulation.
We formulated the control scheme and verified it with
a  planar and a spatial model through numerical simulation.
From the results,
we confirmed that the performance of the task space control is as accurate as
that of a method based on the Operational Space formulation,
which is the most well-known/used method.

We found out that the joint motion under Reaction Null-Space control
coincides with that under kinematic control with minimum acceleration norm solution,
when the constrained link mass/inertia is massive.

As a different issue in redundancy,
we tackled the joint stabilization problem.
It is known that unstable behavior of joint motion appears even when
the task space is stable,
because a method cannot be integrable.
With regard to this problem,
we formulated a velocity-based stabilization term for Reaction Null-Space control.
Then, it was confirmed that the performance is better than that of a simple damping term.


%%%%%%%%%%%%%%%%%%%%%%%%
\section{Future tasks}
%%%%%%%%%%%%%%%%%%%%%%%%
%%%%%%%%%%%%%%%%%%%%%%%%%%%%%%%%%%%%%%%%%%%
\subsubsection{Reactionless motion control}
%%%%%%%%%%%%%%%%%%%%%%%%%%%%%%%%%%%%%%%%%%%
%
\begin{itemize}
\item Singularity of the coupling inertia matrix:
  Even though kinematic singularities have been much discussed,
  singularities resulting from the coupling inertia matrix has been unclear, so far.
  Because the matrix is highly nonlinear and complicated,
  we cannot discuss it analytically.
\item Manipulator design for reactionless motion control:
  In this work, we only dealt with a seven-DoF redundant manipulator.
  With the model, position control of the end-effector under reactionless motion is impossible.
  However, position control of the end-effector can be used in several tasks such as
  assembly, inspection and construction and so on.
  Hence, we should improve the performance of the control method.
  One possible way is making use of appropriate manipulators for execution of reactionless motion control.
  By adding additional DoFs, we can expect that the movable region of the end-effector under reactionless motion
  broadens.
\end{itemize}
%

%%%%%%%%%%%%%%%%%%%%%%%%%%%%%%%%%%%%%%%%%%%
\subsubsection{Motion/Force control}
%%%%%%%%%%%%%%%%%%%%%%%%%%%%%%%%%%%%%%%%%%%
%
\begin{itemize}
\item Influences stemming from the particular solution component:
  We compared the proposed Reaction Null-Space based control with the Operational Space formulation
  based control.
  Then, it was confirmed that the control accuracy in task space was identical,
  while joint space behavior was quite different.
  Even so, if we add a null-space component, the two joint motions become the same one.
  Our concern is whether the particular components make the system dynamics different
  even when a redundant control is executed.
\end{itemize}
%

%**********************************************************************
%
%
%%% Local Variables:
%%% mode: latex
%%% TeX-master: "./main"
%%% End: